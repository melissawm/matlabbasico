\documentclass[hyperref={pdfpagelabels=false}]{beamer}
\usetheme[block=fill]{metropolis}

\usepackage[portuguese]{babel}
\usepackage[utf8]{inputenc} % To use characters such as é without typing é
\usepackage{ctable}
\usepackage{listings}
\lstset{%
  language=Matlab,
  showstringspaces=false,
  basicstyle=\linespread{0.9}\ttfamily,
  keywordstyle=\textbf,
  commentstyle=\color{gray},
  stringstyle=\color{orange},
  numbers=left,
  numberstyle=\tiny\color{gray},
  stepnumber=1,
  numbersep=10pt,
  columns=fullflexible,
  tabsize=3,
  frame=single,
  frameround=tttt}
\let\Tiny=\tiny % eliminates compilation errors
\usepackage{fontspec}

\title{Laboratório de Matemática Computacional I}
\subtitle{Aula 14}
\author[M. Weber Mendonça]{Melissa Weber Mendonça\\
Universidade Federal de Santa Catarina}
\date{2011}

\begin{document}
\setmonofont{Inconsolata}

\begin{frame}
  \titlepage
\end{frame}

\begin{frame}{Aquecimento}
  Escreva um programa que, dada uma matriz e um número, mostre todos os elementos da matriz que forem maiores do que este número e suas localizações.
	\vfill
	\begin{center} \href{listings/elementosmaiores.m}{\underline{\texttt{elementosmaiores.m}}} \end{center}
\end{frame}

\begin{frame}{Exercício}
  Escreva um programa que tome uma matriz quadrada $A$ e retorne sua diagonal principal como um vetor.
	\vfill
	\begin{center} \href{listings/diagonal.m}{\underline{\texttt{diagonal.m}}} \end{center}
\end{frame}

\begin{frame}{Exercício}
  Escreva um programa que construa a matriz zig-zag de dimensão $m$ por $n$ quaisquer.
   
  Exemplo:
  $$\begin{bmatrix}
    1 & 2 & 3 & 4 & 5 & 6 & 7\\
    14 & 13 & 12 & 11 & 10 & 9 & 8\\
    15 & 16 & 17 & 18 & 19 & 20 & 21\\
    28 & 27 & 26 & 25 & 24 & 23 & 22
    \end{bmatrix}_{4\times 7}$$
  \vfill
  \begin{center} \href{listings/zigzag.m}{\underline{\texttt{zigzag.m}}} \end{center}
\end{frame}

\begin{frame}{Exercício}
  Dizemos que uma matriz é \emph{diagonalmente dominante} se cada elemento da sua diagonal principal for, em valor absoluto, maior que a soma de todos os valores absolutos dos outros elementos da mesma linha, ou seja, $A$ é diagonalmente dominante se
  \begin{equation*}
    |a_{ii}| \geq \sum_{j\ne i} |a_{ij}| \mbox{ para todo } i.
  \end{equation*}
	Escreva um programa que identifica se uma matriz é diagonalmente dominante ou não.
  \vfill
  \begin{center} \href{listings/diagonaldominante.m}{\underline{\texttt{diagonaldominante.m}}} \end{center}
\end{frame}

\begin{frame}{Exercício}
  Dizemos que uma matriz inteira $A_{n\times n}$ é uma \emph{matriz de permutação} se em cada linha e em cada coluna houver $n-1$ elementos nulos e um único elemento igual a 1. Escreva um programa que verifique se uma matriz é de permutação.

Exemplo:
\begin{math}
\left[
   \begin{array}{c c c c}
      0&1&0&0\\
      0&0&1&0\\
      1&0&0&0\\
      0&0&0&1
   \end{array}\right]
\end{math}
é de permutação, enquanto
\begin{math}
\left[
   \begin{array}{c c c c}
      0&1&0&0\\
      0&0&1&0\\
      1&0&0&0\\
      0&0&0&2
   \end{array}\right]
\end{math}
não é.
\end{frame}
\end{document}
