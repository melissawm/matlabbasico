\documentclass[hyperref={pdfpagelabels=false}]{beamer}
\usetheme[block=fill]{metropolis}

\usepackage[portuguese]{babel}
\usepackage[utf8]{inputenc} % To use characters such as é without typing é
\usepackage{ctable}
\usepackage{listings}
\lstset{%
  language=Matlab,
  showstringspaces=false,
  basicstyle=\linespread{0.9}\ttfamily,
  keywordstyle=\textbf,
  commentstyle=\color{gray},
  stringstyle=\color{orange},
  numbers=left,
  numberstyle=\tiny\color{gray},
  stepnumber=1,
  numbersep=10pt,
  columns=fullflexible,
  tabsize=3,
  frame=single,
  frameround=tttt}
\let\Tiny=\tiny % eliminates compilation errors
\usepackage{fontspec}

\title{Laboratório de Matemática Computacional I}
\subtitle{Aula 12}
\author[M. Weber Mendonça]{Melissa Weber Mendonça\\
Universidade Federal de Santa Catarina}
\date{2011}

\begin{document}
\setmonofont{Inconsolata}

\begin{frame}
  \titlepage
\end{frame}

\begin{frame}{Relembrando...}
	Na aula passada, vimos que podemos tratar as listas de números em MATLAB como vetores:
	\begin{center}
	  $v_1 = [1,2,3,4], v_2 = [0,1,2,3]$\\
	  $v_1+v_2 = [1+0,2+1,3+2,4+3]$
	\end{center}
	Em outras palavras, não precisamos escrever as operações elemento a elemento!
\end{frame}

\begin{frame}{Mas...}
  E se quisermos realizar operações elemento a elemento?
  
  Exemplo: Escreva um programa que toma dois vetores $u$ e $v$ do mesmo tamanho e retorna um terceiro vetor, contendo em cada entrada $i$ o produto $u(i)*v(i)$.
  \vfill Em MATLAB, \\ \begin{center}{\texttt{u.*v}}\end{center}
\end{frame}

\begin{frame}{Exercício}
  Escrever um programa que, dado um vetor, calcula seu quadrado (elemento a elemento).
  \vfill Em MATLAB, \\ \begin{center}{\texttt{u.\^{}2}}\end{center}
\end{frame}

\begin{frame}{Vetores especiais}
  \begin{itemize}
  \item[] {\texttt{zeros(n,1)}}
  \item[] {\texttt{ones(n,1)}}
  \end{itemize}
  \vfill
  \begin{tabular}{r l}
    Exemplo: &{\texttt{u = 5*ones(3,1)}}\\
    &{\texttt{v = zeros(size(u,1),1)}}\\
    &{\texttt{v - u}}
  \end{tabular}
\end{frame}

\begin{frame}{Exercício}
  Escreva um programa que cria um vetor com 100 elementos, de forma que os elementos deste vetor são os elementos da sequência de Fibonacci.  
	\begin{center} \href{listings/fibo.m}{\underline{\texttt{fibo.m}}} \end{center}
\end{frame}

\begin{frame}{Exercício}
  Escreva um programa que, dadas duas listas, uma com numeros de matricula e outra com idades, encontre a idade de um aluno.
	\vfill
	\begin{tabular}{r r c l}
	  Exemplo: & matriculas &=& [1101, 1002, 1001, 1104, 1103]\\
	  & idades &=& [23, 10, 18, 34, 21]
	\end{tabular}
	\vfill
	Entrada: 1102\\
	Saída: 'A idade do aluno 1102 eh 10 anos.'
	\begin{center} \href{listings/encontraaluno.m}{\underline{\texttt{encontraaluno.m}}} \end{center}
\end{frame}

\begin{frame}{Matrizes}
  Poderíamos ter escrito:
  \vfill
  \begin{minipage}{5cm}
    \begin{tabular}{c c}
			Matrícula & Idade\\\toprule
      1101 & 23\\\midrule
      1002 & 10\\\midrule
      1001 & 18\\\midrule
      1104 & 34\\\midrule
      1103 & 21
    \end{tabular}
	\end{minipage}
	\begin{minipage}{5cm}
    \only<2->{\begin{tabular}{r c l}
	      dados &=& [1101 23;\\
	        & &  1002 10;\\
	        & &  1001 18;\\
	        & &  1104 34;\\
	        & &  1103 21]
	  \end{tabular}}
	\end{minipage}
	\only<3>{
	  \begin{alertblock}{}
		\begin{center} No MATLAB, todos os objetos são matrizes!! \end{center}
	\end{alertblock}}
\end{frame}

\begin{frame}{Exemplo}
	No console, criar uma matriz que contém uma lista de matrículas de alunos e suas notas na primeira e segunda provas.
\end{frame}

\begin{frame}{Exemplo}
  Dada a matriz do exercício anterior, criar um programa que lê uma matrícula de aluno e informa ao usuário suas notas nas duas provas.

	\begin{center} \href{listings/notas.m}{\underline{\texttt{notas.m}}} \end{center}
\end{frame}

\begin{frame}{Exemplo}
  Ainda usando a mesma matriz, modificar seu programa para que, dada a matrícula de um aluno, ele informe ao usuário a nota da prova $j$ (também dada pelo usuário), ou a média do aluno, caso $j=0$.

	\begin{center} \href{listings/notasemedias.m}{\underline{\texttt{notasemedias.m}}} \end{center}
\end{frame}

\begin{frame}{Exemplo}
  Modifique o programa anterior para que a média seja guardada em uma nova coluna da matriz.

	\begin{center} \href{listings/notasemedias2.m}{\underline{\texttt{notasemedias2.m}}} \end{center}
\end{frame}

\begin{frame}{Usando \emph{slicing}!}
   
  Para vetores: \\
  \begin{center}
		{\texttt{v = [3,4,2,1,5];}}\\
		{\texttt{v(1:3) = [3,4,2]}}
  \end{center}
  
	Também podemos fazer \emph{slicing} em uma matriz, mas aqui devemos ter
  cuidado com os índices!
  \vfill
	Exemplo: $A = \left[ \begin{array}{c c c} 1 & 2 & 3\\ 4 & 5 & 6\end{array}
      \right]$\\
	\begin{center}
	  {\texttt{A(1,:)}}\\
	  {\texttt{A(:,2)}}\\
		{\texttt{A(1:2,1:2)}}
	\end{center}
\end{frame}

\begin{frame}{Exemplo}
  Escrever um programa que acrescenta à sua matriz anterior novas matrículas e notas de alunos, e que ao final calcula a média para todos os alunos.
\end{frame}

\end{document}
