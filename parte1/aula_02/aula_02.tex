\documentclass[hyperref={pdfpagelabels=false}]{beamer}
\usetheme[block=fill]{metropolis}

\usepackage[portuguese]{babel}
\usepackage[utf8]{inputenc} % To use characters such as é without typing é
\usepackage{ctable}
\usepackage{listings}
\lstset{%
  language=Matlab,
  showstringspaces=false,
  basicstyle=\linespread{0.9}\ttfamily,
  keywordstyle=\textbf,
  commentstyle=\color{gray},
  stringstyle=\color{orange},
  numbers=left,
  numberstyle=\tiny\color{gray},
  stepnumber=1,
  numbersep=10pt,
  columns=fullflexible,
  tabsize=3,
  frame=single,
  frameround=tttt}
\let\Tiny=\tiny % eliminates compilation errors
\usepackage{fontspec}

\title{Laboratório de Matemática Computacional I}
\subtitle{Aula 2}
\author[M. Weber Mendonça]{Melissa Weber Mendonça\\
Universidade Federal de Santa Catarina} 
\date{2011}

\begin{document}
\setmonofont{Inconsolata}

\begin{frame}
  \titlepage
\end{frame}

\begin{frame}{Problema 1}
  Escrever um programa que pede para o usuário tentar adivinhar um número entre 0 e 10.
\end{frame}

\begin{frame}{If - Else - End: Se - Senão - Fim}
  O If (Se) representa uma sentença lógica condicional: 
  \begin{center}
    \begin{minipage}{0.7\textwidth}
      \begin{itemize}
      \item[] Se (sentença lógica for verdadeira) então
	      \begin{itemize}
	      \item[] faça (1)
	      \end{itemize}
      \item[] Senão
	      \begin{itemize}
	      \item[] faça (2) 
	      \end{itemize}
      \item[] Fim Se 
      \end{itemize}
    \end{minipage}
    \vfill
    \begin{minipage}{0.8\textwidth}
      \begin{alertblock}{}
        Em Matlab, uma sentença lógica pode ter dois valores:\\
        \begin{center} 0 (Falso) ou 1 (Verdadeiro)\end{center}
      \end{alertblock}
    \end{minipage}
  \end{center}
\end{frame}

\begin{frame}{Problema 1}
  Para comparar dois números $a$ e $b$, usamos os seguintes sinais em Matlab:
  \begin{columns}
    \column{5cm} 
    \begin{itemize}
    \item $a$ é igual a $b$? 
    \item $a$ é maior que $b$?
    \item $a$ é menor que $b$?
    \item $a$ é maior ou igual a $b$?
    \item $a$ é menor ou igual a $b$?
    \item $a$ é diferente de $b$?
    \end{itemize}
    \column{5cm}
    \begin{itemize}
    \item[] {\texttt{a == b}}
    \item[] {\texttt{a > b}}
    \item[] {\texttt{a < b}}
    \item[] {\texttt{a >= b}}
    \item[] {\texttt{a <= b}}
    \item[] {\texttt{a $\sim$= b}}
    \end{itemize}
  \end{columns}
\end{frame}

\begin{frame}{Exemplos}
  Testar no console do Matlab:
  \begin{columns}
    \column{5cm}
    \begin{itemize}	
    \item 0 é igual a 1?
    \item 2 é maior que 1?
    \item 1 é igual a 1?
    \item 1 é maior ou igual a 1?
    \item $\sin{(\pi)}$ é igual a 0?
    \end{itemize}
    \column{5cm}
    \setbeamercovered{invisible}
    \begin{itemize}
    \item<2-> \texttt{0 == 1}
    \item<3-> \texttt{2 > 1}
    \item<4-> \texttt{1 == 1}
    \item<5-> \texttt{1 >= 1}
    \item<6-> \texttt{sin(pi) == 0}
    \end{itemize}
  \end{columns}
\end{frame}

\begin{frame}{Problema 1 - Resposta}
   Escrever um programa que pede para o usuário tentar adivinhar um número
 entre 0 e 10.
  \begin{center}
    \begin{minipage}{0.5\textwidth}
      \begin{alertblock}{}
	\begin{center}
	  Aqui, não precisamos usar funções!
	\end{center}
      \end{alertblock}
    \end{minipage}
  \end{center}
\end{frame}

\begin{frame}{Problema 1 - Resposta}
   Escrever um programa que pede para o usuário tentar adivinhar um número
 entre 0 e 10.
   \lstinputlisting[title=\texttt{guess.m}]{listings/guess.m}
\end{frame}

\begin{frame}{Problema 1 - Resposta}
   \lstinputlisting[title=\texttt{guess2.m}]{listings/guess2.m}
\end{frame}

\begin{frame}{Problema 1 - Resposta alternativa}
  \lstinputlisting[title=\texttt{guess3.m}]{listings/guess3.m}
\end{frame}

\begin{frame}{Outros Problemas}
  \setbeamercovered{invisible}
  \only<1>{%
    Escrever um programa que receba um número real qualquer e exiba uma mensagem caso ele seja negativo.
    \begin{center}
      Resposta: \href{listings/negativo.m}{\underline{\texttt{negativo.m}}}
    \end{center}
  }
  \only<2>{%
    Escrever um programa que divide um número $a$ por um número $b$, testando se o denominador é zero.
    \begin{center}
      Resposta: \href{listings/divisao.m}{\underline{\texttt{divisao.m}}}
    \end{center}
  }
  \only<3>{%
    Escrever um programa que determina se um número dado é par ou ímpar.
    \begin{block}{}
      \begin{center}
        Use a função {\texttt{rem(número,divisor)}}
      \end{center}
    \end{block}
    \begin{center}
      Resposta: \href{listings/parouimpar.m}{\underline{\texttt{parouimpar.m}}}
    \end{center}
  }
\end{frame}

\begin{frame}{ElseIf}
   Nos casos em que precisamos de mais de 2 alternativas, podemos usar um atalho: {\texttt{elseif}}
   \begin{center}
     \begin{minipage}{0.7\textwidth}
       \begin{itemize}
       \item[] Se (sentença lógica) então
	       \begin{itemize}
	       \item[] faça (1)
	       \end{itemize}
       \item[] SenãoSe
	       \begin{itemize}
	       \item[] faça (2) 
	       \end{itemize}
		   \item[] Senão
		     \begin{itemize}
		     \item[] faça (3)
		     \end{itemize}
       \item[] Fim Se 
       \end{itemize}
     \end{minipage}
   \end{center}
\end{frame}

\begin{frame}{ElseIf: exemplo 1}
  Escrever um programa que, dadas as notas de 3 provas obtidas por um aluno no semestre, retorne sua média (calculada pela média aritmética das 3 provas) e diga se o aluno foi aprovado, reprovado ou se está em recuperação.
  \begin{center}
    \begin{minipage}{0.8\textwidth}
	    \only<1>{\lstinputlisting[title={\texttt{elif2-if.m}}]{listings/elif2-if.m}}
      \only<2>{\lstinputlisting[title={\texttt{elif2.m}}]{listings/elif2.m}}
    \end{minipage}
  \end{center}
\end{frame}

\begin{frame}{ElseIf: exemplo 2}
  Escrever um programa que, dada a idade de uma pessoa, caracteriza essa pessoa como criança, adolescente, adulto ou idoso.
\end{frame}

\begin{frame}{ElseIf: resposta}
  \begin{center}
    \begin{minipage}{0.85\textwidth}
      \lstinputlisting[basicstyle=\footnotesize, title=\texttt{elif.m}]{listings/elif.m}
    \end{minipage}
  \end{center}
\end{frame}

\begin{frame}{e - \texttt{\&\&}}
  Muitas vezes, precisamos agrupar valores lógicos:
  
  \begin{block}{}
		Se uma pessoa for homem \alert{e} tiver mais de 18 anos, deve se inscrever no serviço militar.
	\end{block}
	\only<1>{
	  \begin{itemize}
	  \item[] Se gênero {\texttt{==}} homem então
	    \begin{itemize}
			\item[] Se idade {\texttt{>=}} 18 então
			  \begin{itemize}
				\item[] Inscreva-se no serviço militar.
			  \end{itemize}
			\item[] Senão
			  \begin{itemize}
				\item[] Você está liberado!
			  \end{itemize}
			\item[] Fim Se
	    \end{itemize}
	  \item[] Senão
	    \begin{itemize}
			\item[] Você está liberade!
	    \end{itemize}
	  \item[] Fim Se
	  \end{itemize}
  }
	\only<2>{
	  \begin{itemize}
	  \item[] Se gênero {\texttt{==}} homem \alert{e} idade {\texttt{>=}} 18 então
			\begin{itemize}
			\item[] Inscreva-se no serviço militar.
			\end{itemize}
		\item[] Senão
			\begin{itemize}
			\item[] Você está liberado!
			\end{itemize}
		\item[] Fim Se
	  \end{itemize}
  }
	\only<3>{
	  \begin{itemize}
	  \item[] Se \alert{(}gênero {\texttt{==}} homem\alert{)} \alert{e} \alert{(}idade {\texttt{>=}} 18\alert{)} então
			\begin{itemize}
			\item[] Inscreva-se no serviço militar.
			\end{itemize}
		\item[] Senão
			\begin{itemize}
			\item[] Você está liberado!
			\end{itemize}
		\item[] Fim Se
	  \end{itemize}
  }
\end{frame}

\begin{frame}{e - Tabela Verdade}
  \setbeamercovered{invisible}
  \begin{center}
  \begin{tabular}{c c c c}
     \texttt{a} & \texttt{b} & & \texttt{a \&\& b}\\\toprule
     \only<1->{Falso} & \only<1->{Falso} & & \only<2->{Falso}\\\midrule
     \only<3->{Verdadeiro} & \only<3->{Falso} & &\only<4->{Falso}\\\midrule
     \only<5->{Falso} & \only<5->{Verdadeiro} & &\only<6->{Falso}\\\midrule
     \only<7->{Verdadeiro} & \only<7->{Verdadeiro} & &\only<8->{Verdadeiro}\\\bottomrule
  \end{tabular}
  \end{center}
\end{frame}

\begin{frame}{e - Tabela Verdade (Matlab)}
	\begin{center}
  \begin{tabular}{c c c c}
     \texttt{a} & \texttt{b} & & \texttt{a \&\& b}\\\toprule
     0 & 0 & & 0\\\midrule
     1 & 0 & &0\\\midrule
     0 & 1 & &0\\\midrule
     1 & 1 & &1\\\bottomrule
  \end{tabular}
  \end{center}
\end{frame}

\begin{frame}{e - \texttt{\&\&}}
   No Matlab: representamos o \alert{e} por \alert{{\texttt{\&\&}}}:
   \lstinputlisting[title=\texttt{exemploe.m}]{listings/exemploe.m}
\end{frame}

\begin{frame}
   \frametitle{e - \texttt{\&\&}: Exemplos}
   \only<1>{%
     Escrever um programa que decida se um número está em um intervalo dado $[a,b]$.
     \begin{center}
       Resposta: \href{listings/intervalo.m}{\underline{\texttt{intervalo.m}}}
     \end{center}
   }
   \only<2>{%
     Escrever um programa que decida se um número é divisível por 6.
   }
\end{frame}
\end{document}
