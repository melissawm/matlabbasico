\documentclass[hyperref={pdfpagelabels=false}]{beamer}
\usetheme[block=fill]{metropolis}

\usepackage[portuguese]{babel}
\usepackage[utf8]{inputenc} % To use characters such as é without typing é
\usepackage{ctable}
\usepackage{listings}
\lstset{%
  language=Matlab,
  showstringspaces=false,
  basicstyle=\linespread{0.9}\ttfamily,
  keywordstyle=\textbf,
  commentstyle=\color{gray},
  stringstyle=\color{orange},
  numbers=left,
  numberstyle=\tiny\color{gray},
  stepnumber=1,
  numbersep=10pt,
  columns=fullflexible,
  tabsize=3,
  frame=single,
  frameround=tttt
}
\let\Tiny=\tiny % eliminates compilation errors
\usepackage{fontspec}

\title{Laboratório de Matemática Computacional II}
\subtitle{Aula 6}
\author[M. Weber Mendonça]{Melissa Weber Mendonça\\
Universidade Federal de Santa Catarina}
\date{2011}

\begin{document}
\setmonofont{Inconsolata}

\begin{frame}
  \titlepage
\end{frame}

\begin{frame}{Anteriormente...}
  Estrutura \emph{célula}:
  \begin{center}
    \texttt{tabela = \{ 'Maria', 17, [12, 14] \}}
  \end{center}
\end{frame}

\begin{frame}{Aquecimento}

  Escreva um programa que preencha uma tabela de nomes de funcionários e quantas horas este funcionário trabalhou na semana (entre 8 e 40 horas) e calcule o total da folha de pagamento semanal, sendo que a empresa paga 25 reais por hora trabalhada.

  \begin{center}
    \href{listings/folha.m}{\underline{\texttt{folha.m}}}
  \end{center}
\end{frame}

\begin{frame}{Exercício}

  Escreva um programa que receba uma lista de nomes e sobrenomes, e ordene a lista por ordem alfabética de nomes.

  Atenção: para acessar cada elemento de um vetor (ou de um texto!) armazenado dentro de uma célula, devemos usar a seguinte sintaxe:

  \begin{center}
    \texttt{>> tabela\{1, 3\}(2)}
  \end{center}

  (acessa o segundo elemento do vetor armazenado na posição (1,3) da célula.)

  \begin{center}
    \href{listings/nomes.m}{\underline{\texttt{nomes.m}}}
  \end{center}
\end{frame}

\begin{frame}{Exercício}
  Escreva um programa que construa um banco de dados de alimentos e preços por unidade, e que calcule o preço de uma compra informada pelo usuário.

  Exemplo de entrada:
  \vfill
  ``Entre com o alimento: {\texttt{laranja}}''
  
  ``Entre com o número de unidades do alimento que deseja comprar: {\texttt{2}}''
  \vfill
  \begin{center}
    \href{listings/compra.m}{\underline{\texttt{compra.m}}}
  \end{center}
\end{frame}

\begin{frame}{Outras possibilidades}

  Podemos construir células contendo outras células:

  \begin{tabular}{r l}
    \texttt{>> v = \{} & \texttt{\alert{\{} 1, 'teste', [1:2] \alert{\}};}\\
    & \texttt{\alert{\{} [0,3], 12, 'nome', rand(4,4) \alert{\}} \}}
  \end{tabular}

  Nesse caso, os elementos devem ser referenciados da seguinte forma:

  \texttt{>> v\{2\}\{1\}}
\end{frame}

\begin{frame}{Exercício}

  Modifique o programa anterior (dos nomes e sobrenomes) para que ele ordene a lista por ordem alfabética de nomes ou sobrenomes, conforme selecionado pelo usuário.

  \vfill

  Sugestão: no campo ``nome'', fazer uma célula com dois elementos: o nome, e o sobrenome.

  \vfill

  \begin{center}
    \href{listings/nomesousobrenomes.m}{\underline{\texttt{nomesousobrenomes.m}}}
  \end{center}
  
\end{frame}

\begin{frame}{Mais comandos}

  Podemos, analogamente ao que fizemos com vetores, concatenar células:

  \begin{itemize}
  \item[\texttt{>>}] \texttt{C1 = \{ 'Joao', 16 \}}
  \item[\texttt{>>}] \texttt{C2 = \{ 'Maria', 18; 'Ricardo', 13 \}}
  \item[\texttt{>>}] \texttt{cola = \{ C1 C2 \}} \qquad \alert{CUIDADO!}
  \item[\texttt{>>}] \texttt{uniao = [ C1\alert{;} C2 ]}
  \end{itemize}
  
\end{frame}

\begin{frame}{Exercício}
  
  Usando a tabela do exercício anterior, acrescente uma ``coluna'' à célula que informa a data de nascimento de cada pessoa.

  \vfill

  \begin{center}
    \href{listings/coluna.m}{\underline{\texttt{coluna.m}}}
  \end{center}

\end{frame}

\begin{frame}{Exercícios}
  
  Escreva um programa que calcule o preço de um computador como soma do custo de cada componente.

  \footnotesize{%
  \begin{itemize}
  \item Placa-mãe: R\$800,00 (obrigatória em todos os computadores)
  \item Opções de memória: 2, 4 ou 8 Gb, cada 2Gb custa R\$150,00.
  \item Opções de processadores:
    \begin{itemize}
    \item Dual core: R\$530,00
    \item Quad core: R\$710,00
    \end{itemize}
  \item Oções de disco rígido:
    \begin{itemize}
    \item 250Gb: R\$200,00
    \item 500Gb: R\$320,00
    \item 1Tb: R\$400,00
    \end{itemize}
  \item Opções de monitor:
    \begin{itemize}
    \item 19 polegadas: R\$320,00
    \item 21 polegadas: R\$520,00
    \end{itemize}
  \end{itemize}
}

\begin{center}
  \href{listings/computadores.m}{\underline{\texttt{computadores.m}}}
\end{center}
  
\end{frame}

\begin{frame}{Outras possiblidades...}

  Vetor de células:
  
  \begin{tabular}{rl}
    \texttt{>> v = [} & \texttt{\alert{\{} 1, 'teste', [1:2] \alert{\}};}\\
                      & \texttt{\alert{\{} [0,3], 12, 'nome' \alert{\}} ]}
  \end{tabular}
  \vfill

  Cuidado: aqui, as células devem ter a mesma dimensão para ficarem em um vetor: se cada célula é 3$\times $ 2, e temos 10 células, o tamanho do vetor de células será 30$\times$2.

  Neste caso, os elementos devem ser referenciados da seguinte forma:

  \texttt{>> v(2)\{1\}}
\end{frame}

\begin{frame}{Revisão}
  \begin{itemize}
  \item \texttt{if} \textemdash\ \texttt{for} \textemdash\ \texttt{while} (Estruturas de repetição e condicionais)
  \item Textos
  \item Vetores
  \item Matrizes
  \item Gráficos
  \item Células
  \end{itemize}
\end{frame}

\end{document}
