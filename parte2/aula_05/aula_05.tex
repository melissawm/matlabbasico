\documentclass[hyperref={pdfpagelabels=false}]{beamer}
\usetheme[block=fill]{metropolis}

\usepackage[portuguese]{babel}
\usepackage[utf8]{inputenc} % To use characters such as é without typing é
\usepackage{ctable}
\usepackage{listings}
\lstset{%
  language=Matlab,
  showstringspaces=false,
  basicstyle=\linespread{0.9}\ttfamily,
  keywordstyle=\textbf,
  commentstyle=\color{gray},
  stringstyle=\color{orange},
  numbers=left,
  numberstyle=\tiny\color{gray},
  stepnumber=1,
  numbersep=10pt,
  columns=fullflexible,
  tabsize=3,
  frame=single,
  frameround=tttt
}
\let\Tiny=\tiny % eliminates compilation errors
\usepackage{fontspec}

\title{Laboratório de Matemática Computacional II}
\subtitle{Aula 5}
\author[M. Weber Mendonça]{Melissa Weber Mendonça\\
Universidade Federal de Santa Catarina}
\date{2011}

\begin{document}
\setmonofont{Inconsolata}

\begin{frame}
  \titlepage
\end{frame}

\begin{frame}{Anteriormente...}
  Vimos que, no MATLAB, podemos resolver um sistema linear 
  $$Ax=b$$
  através do comando
  \begin{itemize}
  \item[\texttt{>>}] \texttt{x = A\textbackslash b}
  \end{itemize}
  \vfill
  Exemplo: Resolva o sistema
  $$
  \begin{pmatrix}
    2 & 1 & 3\\  
    2 & 6 & 8\\
    6 & 8 & 18\\
  \end{pmatrix}
  \begin{pmatrix}
    x_1\\
    x_2\\
    x_3
  \end{pmatrix}
  = 
  \begin{pmatrix}
    1\\
    3\\
    5
  \end{pmatrix}
  $$
  \vfill
  \begin{itemize}
  \item[\texttt{>>}] \texttt{A = [2 1 3;2 6 8; 6 8 18]}
  \item[\texttt{>>}] \texttt{b = [1;3;5]}
  \item[\texttt{>>}] \texttt{x = A\textbackslash b}
  \end{itemize}
\end{frame}

\begin{frame}{Sistema triangular inferior}
  \only<1>{%
    \begin{equation*}
      \begin{pmatrix}
        2 & 0 & 0\\  
        2 & 6 & 0\\
        6 & 8 & 18\\
      \end{pmatrix}
      \begin{pmatrix}
        x_1\\
        x_2\\
        x_3
      \end{pmatrix}
      = 
      \begin{pmatrix}
        1\\
        3\\
        5
      \end{pmatrix}
    \end{equation*}
    \begin{align*}
      2x_1 = 1 &\Rightarrow x_1 = 1/2\\
      2x_1+6x_2 = 3 &\Rightarrow x_2 = (3-2x_1)/6\\
      6x_1+8x_2+18x_3 = 5 &\Rightarrow x_3 = (5-6x_1-8x_2)/18
    \end{align*}
  }
  \only<2>{%
    Em geral, $$x_i = \frac{(b_i-(a_{i1}x_1+a_{i2}x_2+\ldots + a_{i,i-1}x_{i-1}))}{a_{ii}}$$
    Para $i=1$ até $n$, faça $$x_i = \frac{\left(b_i -\displaystyle \sum_{j=1}^{i-1} a_{ij}x_j\right)}{a_{ii}}$$
    \vfill
    \begin{center}
      \href{listings/sistematriangularinferior.m}{\underline{\texttt{sistematriangularinferior.m}}}
    \end{center}
    }
\end{frame}

\begin{frame}{Sistema triangular superior}
  \only<1>{%
    \begin{equation*}
      \begin{pmatrix}
        2 & 1 & 3\\  
        0 & 6 & 8\\
        0 & 0 & 18\\
      \end{pmatrix}
      \begin{pmatrix}
        x_1\\
        x_2\\
        x_3
      \end{pmatrix}
      = 
      \begin{pmatrix}
        1\\
        3\\
        5
      \end{pmatrix}
    \end{equation*}
    \begin{align*}
      18x_3 = 5 &\Rightarrow x_3 = 5/18\\
      6x_2+8x_3 = 3 &\Rightarrow x_2 = (3-8x_3)/6\\
      2x_1+x_2+3x_3 = 1 &\Rightarrow x_1 = (1-x_2-3x_3)/2
    \end{align*}
  }
  \only<2>{%
    Em geral, $$x_i = \frac{(b_i-(a_{i,i+1}x_{i+1} +a_{i,i+2}x_{i+2}+\ldots + a_{in}x_n))}{a_{ii}}$$
    Para $i=n$ até $1$, faça $$x_i = \frac{\left(b_i -\displaystyle \sum_{j=i+1}^n a_{ij}x_j\right)}{a_{ii}}$$
    \vfill
    \begin{center}
      \href{listings/sistematriangularsuperior.m}{\underline{\ttfamily{sistematriangularsuperior.m}}}
    \end{center}
    }
\end{frame}

\begin{frame}{Escalonamento - Eliminação Gaussiana}
   O processo de escalonamento visa transformar uma matriz qualquer $A\in {\mathbb{R}}^{n\times n}$ em uma matriz triangular superior:
   $$A = \begin{pmatrix}
     a_{11} & a_{12} & \cdots & a_{1n}\\
     a_{21} & a_{22} & \cdots & a_{2n}\\
     \vdots & & \ddots & \vdots \\
     a_{n1} & \cdots & 0 & a_{nn}
   \end{pmatrix}
   \rightarrow U = 
   \begin{pmatrix}
     u_{11} & u_{12} & \cdots & u_{1n}\\
     0 & u_{22} & \cdots & u_{2n}\\
     \vdots & & \ddots & \vdots \\
     0 & \cdots & 0 & u_{nn}
   \end{pmatrix}.$$
   Para fazer isso, efetuamos operações nas linhas da matriz $A$.

   Como zerar os elementos abaixo de $a_{ii}$ (pivô)?

   Para $k=i+1$ até $n$, \begin{center}(linha $k$) - $\dfrac{a_{ki}}{a_{ii}}$(linha $i$)\end{center}
\end{frame}

\begin{frame}{Eliminação Gaussiana}
   Escreva um programa que transforme uma matriz quadrada
   $$A \in {\mathbb{R}}^{n\times n}$$
   em uma matriz triangular superior $U$ usando a eliminação Gaussiana.
   \vfill
   \begin{center}
     \href{listings/eliminacaogaussiana.m}{\underline{\texttt{eliminacaogaussiana.m}}}
   \end{center}
\end{frame}

\begin{frame}{Eliminação Gaussiana - Sistemas}
   Considere o sistema 
   $$Ax=b, A\in {\mathbb{R}}^{n\times n}, x,b\in {\mathbb{R}}^n.$$

   Escreva um programa que resolva o sistema acima usando eliminação Gaussiana.
   \begin{itemize}
      \item[] \texttt{for j = k+1:n}
      \item[] 
      \begin{itemize}
         \item[] \texttt{multiplicador = U(j,k)$/$U(k,k);}
         \item[] \texttt{U(j,:) = U(j,:) - multiplicador*U(k,:);}
      \end{itemize}
      \item[] \texttt{end}
   \end{itemize}
   \begin{center}
     \href{listings/sistemaporeliminacao.m}{\underline{\texttt{sistemaporeliminacao.m}}}
   \end{center}
\end{frame}

\begin{frame}{Fatoração $A=LU$}
   Sabemos que, se uma matriz for inversível, ela pode ser decomposta em 
   $$A=LU$$
   onde $L$ é uma matriz triangular inferior (com diagonal 1) e $U$ é uma matriz triagular superior. 

   Como encontrar $L$ e $U$ através da eliminação gaussiana?

   Sabemos que:

   \begin{itemize}
      \item[] \texttt{for j = k+1:n}
      \item[] 
      \begin{itemize}
         \item[] \texttt{multiplicador = U(j,k)$/$U(k,k);}
         \item[] \alert{\texttt{L(j,k) = multiplicador;}}
         \item[] \texttt{U(j,:) = U(j,:) - multiplicador*U(k,:);}
      \end{itemize}
      \item[] \texttt{end}
   \end{itemize}
   \begin{center}
     \href{listings/fatoracaolu.m}{\underline{\texttt{fatoracaolu.m}}}
   \end{center}
\end{frame}

\begin{frame}{Troca de linhas}
  \only<1>{Se encontrarmos um pivô nulo durante a eliminação, ainda podemos tentar trocar as linhas de lugar e continuar a eliminação:}

  Exemplo: 
  $$A = \begin{pmatrix} 1 & 1 & 1 \\ 1 & 1 & 3 \\ 2 & 5 & 8 \end{pmatrix} \rightarrow \begin{pmatrix} 1 & 1 & 1 \\ 0 & 0 & 2 \\ 0 & 3 & 6 \end{pmatrix}\rightarrow \alert{\begin{pmatrix} 1 & 1 & 1 \\ 0 & 3 & 6\\0 & 0 & 2  \end{pmatrix}}$$
  
  Uma matriz de permutação é uma $P$, permutação da identidade: neste caso,
  $$ P = \begin{pmatrix} 1 &0&0\\ 0&0&1\\0&1&0\end{pmatrix}$$
  \only<2>{Com isso, podemos determinar uma fatoração $PA=LU$, em que $P$ acumula todas as trocas de linhas ($k$) necessárias à eliminação:
    $$P = P_{k}P_{k-1}\ldots P_2 P_1$$}
\end{frame}

\begin{frame}{$PA=LU$}
   \begin{center}
     \href{listings/palu.m}{\underline{\texttt{palu.m}}}
   \end{center}
   Para testar: use os comandos abaixo para gerar um sistema que tem solução $(1,1,\ldots,1) \in {\mathbb{R}}^n$.
   \begin{itemize}
      \item[\ttfamily{>>}] \ttfamily{A = rand(m,n);}
      \item[\ttfamily{>>}] \ttfamily{b = A*ones(n,1);}
   \end{itemize}
\end{frame}

\begin{frame}{Estruturas de dados Heterogêneas}

   Muitas vezes, gostaríamos de armazenar dados da seguinte forma:
   \begin{center}
     \footnotesize{%
       \begin{tabular}{l c c}
         Título & Núm. Páginas & Datas de Empréstimo e Devolução\\\toprule
         ``Álgebra Linear'' & 205 & 12/08, 15/08\\
         ``Cálculo'' & 346 & 10/09, 12/09\\
         ``Geometria'' & 123 & 04/08, 05/09\\
         ``Topologia'' & 253 & 01/08, 04/09
       \end{tabular}
       }
   \end{center}

   Porém, estes dados são de natureza \emph{heterogênea}: misturamos texto (string), números e intervalos. Como armazenar isso em uma só tabela no MATLAB?
\end{frame}

\begin{frame}{Estrutura \emph{Cell}}
   No MATLAB, podemos fazer o seguinte:
   \begin{itemize}
      \item[\ttfamily{>>}] \ttfamily{tabela = \{ 'Algebra Linear', 205, [1208, 1508];}
      \item[\ttfamily{>>}] \ttfamily{\hskip2.2cm 'Calculo', 346, [1009,1209];}
      \item[\ttfamily{>>}] \ttfamily{\hskip2.2cm 'Geometria', 123, [0408,0509];}
      \item[\ttfamily{>>}] \ttfamily{\hskip2.2cm 'Topologia', 253, [0108,0409] \} }
   \end{itemize}
   A célula funciona como uma matriz, mas aqui os índices são dados sempre entre chaves: \ttfamily{\{\}}.
\end{frame}

\begin{frame}{Comandos}
   Para ver o que está armazenado na variável \ttfamily{tabela}, basta usarmos o comando
   \begin{itemize}
      \item[\ttfamily{>>}] \ttfamily{celldisp(tabela)}
   \end{itemize}
   Para verificar o tamanho de uma célula, usamos o comando
   \begin{itemize}
      \item[\ttfamily{>>}] \ttfamily{size(tabela)}
   \end{itemize}
   Para criar uma célula vazia com $m$ por $n$ elementos, usamos o comando
   \begin{itemize}
      \item[\ttfamily{>>}] \ttfamily{tabela = cell(m,n)}
   \end{itemize}
   Para apagar um elemento da célula, podemos usar a notação de matriz vazia:
   \begin{itemize}
      \item[\ttfamily{>>}] \ttfamily{tabela\{1,2\} = []}
   \end{itemize}
\end{frame}

\begin{frame}{Exercício}
   Escreva um programa que receba uma lista de nomes e datas de nascimento, e um mês. Seu programa deve dizer quais pessoas fazem aniversário naquele mês.

   Exemplo:\\
   \begin{center}
   \begin{tabular}{l l}
      Nome & Mês\\\toprule
      'Joao' & 'setembro'\\
      'Maria' & 'janeiro'\\
      'Roberta' & 'julho'\\
      'Danilo' & 'dezembro'
   \end{tabular}
   \end{center}
   \vfill
   \begin{center}
     \href{listings/aniversarios.m}{\underline{\ttfamily{aniversarios.m}}}
   \end{center}
\end{frame}

\begin{frame}{Exercício}
   Escreva um programa que receba uma lista de matrizes, e suas duas dimensões ($m$ e $n$). No final, diga quais matrizes podem ser multiplicadas e em que ordem.
   \vfill
   \begin{center}
     \href{listings/podemmultiplicar.m}{\underline{\ttfamily{podemmultiplicar.m}}}
   \end{center}
\end{frame}
\end{document}
